\hypertarget{repository-structure}{%
\subsection*{Repository Structure}\label{repository-structure}}

\begin{verbatim}
Bio-Plausible-DeepLearning
+- workspace.ipynb: Main workspace notebook - Execute for replication
\end{verbatim}

\hypertarget{how-to-use-this-code}{%
\subsection*{How to use this code}\label{how-to-use-this-code}}

\begin{enumerate}
\def\labelenumi{\arabic{enumi}.}
\tightlist
\item
  Clone the repo.
\end{enumerate}

\begin{verbatim}
git clone https://github.com/RobertTLange/Bio-Plausible-DeepLearning
cd Bio-Plausible-DeepLearning
\end{verbatim}

\begin{enumerate}
\def\labelenumi{\arabic{enumi}.}
\setcounter{enumi}{1}
\tightlist
\item
  Create a virtual environment (optional but recommended).
\end{enumerate}

\begin{verbatim}
virtualenv -p python BPDL
\end{verbatim}

Activate the env (the following command works on Linux, other operating
systems might differ):

\begin{verbatim}
source BPDL/bin/activate
\end{verbatim}

\begin{enumerate}
\def\labelenumi{\arabic{enumi}.}
\setcounter{enumi}{2}
\tightlist
\item
  Install all dependencies:
\end{enumerate}

\begin{verbatim}
pip install -r requirements.txt
\end{verbatim}

\begin{enumerate}
\def\labelenumi{\arabic{enumi}.}
\setcounter{enumi}{3}
\tightlist
\item
  Run the main notebook:
\end{enumerate}

\begin{verbatim}
jupyter notebook workspace.ipynb
\end{verbatim}
